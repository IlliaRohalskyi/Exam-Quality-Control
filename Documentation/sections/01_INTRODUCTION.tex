\pagenumbering{roman}
\newpage
\tableofcontents


\newpage
\pagenumbering{arabic} 

\section{Introduction}
\subsection{Context of the Work}
The "Exam Quality Control" program aims to address the challenges faced in managing and evaluating exam schedules in educational institutions and organizations. Creating effective and fair exam schedules while following various rules and constraints can be difficult. Manual evaluation of exam plans is time-consuming and prone to mistakes. Therefore, it is essential to develop an automated software solution that can examine exam schedules based on predefined rules.
\subsection{Motivation for This Work}
The motivation behind developing the "Exam Quality Control" program is to evaluate the current exam schedule with utmost accuracy and efficiency. The main objective is to identify any conflicts between the evaluated exam plan and the specified rules and assist in creating the subsequent exam schedule that adheres as closely as possible to these rules. By automating this process, our institution can simplify their scheduling procedures, reduce conflicts, and improve the overall quality of exams. This software gives administrators and schedulers a reliable tool to follow regulations, allocate resources effectively, and provide the best possible exam experience for both students and examiners.
\subsection{Objective for This Work}
The primary objective of the "Exam Quality Control" program is to provide a comprehensive and automated solution for evaluating exam schedules based on predefined rules. The software will consider multiple rules during the evaluation process, including:\vspace{\baselineskip}

    \textbf{Big Exams Early:} \vspace{\baselineskip}
    
    \textbf{Per Exam One Day:} \vspace{\baselineskip}
    
    \textbf{Room Capacity:}\vspace{\baselineskip}
    
    \textbf{Room Distances:} 
\vspace{\baselineskip}
    

By evaluating exam schedules based on these rules and potentially additional ones, the software will provide scores for each rule, enabling administrators to identify areas for improvement and make data-driven decisions to optimize the overall exam plan. The objective is to simplify the scheduling process, enhance the quality of exam plans, and improve the overall experience for both students and examiners.

\vspace{\baselineskip}
