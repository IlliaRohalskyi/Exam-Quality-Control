\section{Retrospective}
\subsection{Evaluation of the project results}

The team's assessment of the project is that while it is not considered fully complete, it showcases promising potential for further enhancements, as mentioned earlier. However, the inner logic of the application is well-designed and demonstrates a high level of quality. The project successfully incorporates multiple evaluation criteria and effectively models them in an efficient and robust manner. This accomplishment represents a significant step forward in providing a viable solution to the problem of exam scoring.

\vspace{\baselineskip}

The successful implementation of the evaluation criteria demonstrates the team's ability to handle complex requirements and deliver a functional solution. The application's capacity to assess exam plans based on predefined criteria contributes to improving the quality and fairness of the evaluation process. The team's effort in designing and implementing the inner logic of the application is commendable.

\vspace{\baselineskip}


Moving forward, the team acknowledges the areas for improvement highlighted earlier, such as enhancing the scoring system, improving installability, and developing a graphical user interface. By addressing these aspects, the application can be further refined to offer an even better user experience and provide more accurate evaluation results.


\vspace{\baselineskip}


Overall, the team's evaluation of the project is positive, recognizing the solid foundation and potential for future improvements. The successful implementation of the inner logic and the incorporation of multiple evaluation criteria demonstrate the team's competence and commitment to delivering a valuable solution to the problem of exam scoring.

\subsection{Project management method and tools used}

The project team employed the Agile and Scrum methodologies for project management. These methodologies provided a suitable framework for managing the workload while maintaining flexibility in terms of work schedules. Here are some details regarding the project management method and tools used:

\vspace{\baselineskip}

For our project, we used Agile and Scrum methodologies. Agile principles emphasize flexibility, collaboration, and iterative development. Scrum, a specific Agile framework, was adopted to organize the work into time-boxed iterations (sprints) and facilitate regular communication and feedback.

\vspace{\baselineskip}

The team members worked individually or in pairs, depending on the task requirements. This approach allowed for a balanced distribution of workload and encouraged collaboration and knowledge sharing among team members.


\vspace{\baselineskip}

The team held regular meetings to ensure effective communication and progress tracking. This included weekly meetings with the supervisor to receive guidance, feedback, and align project goals. Additionally, weekly team meetings were conducted to discuss implementation details, and address any challenges or concerns.

\vspace{\baselineskip}

For the collaboration, we were having a WhatsApp group chat. Maybe it was unprofessional from us, but that made it very easy to collaborate and discuss our project. By doing it in a regular messenger, we established efficient and quick way to communicate. For the version control system, we used BitBucket, which was offered to us in the very beginning of the project.