\section{Retrospective}
\subsection{Evaluation of the project results}

The team's assessment of the project is that while it is not considered fully complete, it showcases promising potential for further enhancements, as mentioned earlier. However, the inner logic of the application is well-designed and demonstrates a high level of quality. The project successfully incorporates multiple evaluation criteria and effectively models them in an efficient and robust manner. This accomplishment represents a significant step forward in providing a viable solution to the problem of exam scoring.

\vspace{\baselineskip}

The successful implementation of the evaluation criteria demonstrates the team's ability to handle complex requirements and deliver a functional solution. The application's capacity to assess exam plans based on predefined criteria contributes to improving the quality and fairness of the evaluation process. The team's effort in designing and implementing the inner logic of the application is commendable.

\vspace{\baselineskip}


Moving forward, the team acknowledges the areas for improvement highlighted earlier, such as enhancing the scoring system, improving installability, and developing a graphical user interface. By addressing these aspects, the application can be further refined to offer an even better user experience and provide more accurate evaluation results.


\vspace{\baselineskip}


Overall, the team's evaluation of the project is positive, recognizing the solid foundation and potential for future improvements. The successful implementation of the inner logic and the incorporation of multiple evaluation criteria demonstrate the team's competence and commitment to delivering a valuable solution to the problem of exam scoring.

\subsection{Project management method and tools used}

The project team employed the Agile and Scrum methodologies for project management. These methodologies provided a suitable framework for managing the workload while maintaining flexibility in terms of work schedules. Here are some details regarding the project management method and tools used:

\vspace{\baselineskip}

For our project, we used Agile and Scrum methodologies. Agile principles emphasize flexibility, collaboration, and iterative development. Scrum, a specific Agile framework, was adopted to organize the work into time-boxed iterations (sprints) and facilitate regular communication and feedback.

\vspace{\baselineskip}

The team members worked individually or in pairs, depending on the task requirements. This approach allowed for a balanced distribution of workload and encouraged collaboration and knowledge sharing among team members.


\vspace{\baselineskip}

The team held regular meetings to ensure effective communication and progress tracking. This included weekly meetings with the supervisor to receive guidance, feedback, and align project goals. Additionally, weekly team meetings were conducted to discuss implementation details, and address any challenges or concerns.

\vspace{\baselineskip}

For the collaboration, we were having a WhatsApp group chat. Maybe it was unprofessional from us, but that made it very easy to collaborate and discuss our project. By doing it in a regular messenger, we established efficient and quick way to communicate. For the version control system, we used BitBucket, which was offered to us in the very beginning of the project.

\subsection{Description of the project process}

The project process began with a kick-off session, followed by the implementation of "BigExamsEarly" over the course of approximately three sprints. Initially, the team encountered challenges in modeling the rules and started with a naive approach, but ended up with an advanced way of handling the rule. As more rules were identified, the team made a decision not to spend excessive time on implementing a perfect solution but focused on implementing each rule incrementally.


\vspace{\baselineskip}

During the subsequent sprints, the team worked on one rule at a time. However, it was later realized that task splitting among team members proved to be difficult, prompting a shift to working on two rules simultaneously. The duration of sprints varied, typically ranging from one to two weeks.


\vspace{\baselineskip}

In the middle of the project, the team recognized the need to rework the code structure and organization due to its initial lack of organization. This restructuring phase took around 2-3 weeks to complete. It is worth noting that during an international week, all team members participated in external activities, resulting in no project work during that period. The team resumed working on the project the following week.

\vspace{\baselineskip}


Towards the end of the project, the remaining rules, specifically those related to professors, were implemented. Extensive testing was performed, and final adjustments were made to the code. Due to time constraints, the decision was made to conclude the project at this stage.


\subsection{Evaluation of one's own way of working and cooperation in the team}

\subsubsection{Illia Rohalskyi}

As a person that was having good knowledge of Python, I was having lots of questions in my team. At some point, it felt like I was taking a leadership role, since I was helping out each of the team members, answering their questions and pushing them to fight procrastination and finish the task before deadline. I found it to be difficult managing people, as everyone has their own way of thinking and you have to find individual way to interact with people. It came down to even explaining the same concept in different ways. It felt like speaking separate languages. But I think I did a good job, as my team was grasping what I was explaining to them. There were technical difficulties due to different backgrounds. Me and Kemal Eren knew how to do linear regression and we proposed lots of ways of doing evaluation criterias, but we were not as proficient in object oriented programming and organising the project as Kaan and Muberra were. In the end, I believe, because we were very diverse, it turned out to be a very juiceful blend of our knowledge. 


\subsubsection{Kaan Özer}


There were a few challenges. First of all, gathering every team member for a meeting was very difficult because each member had different exams, responsibilities, and needed to go to other cities. However, we solved this issue by discussing the days that suited everyone, meeting at the university to prevent distance problems, and sometimes having online meetings. 

\vspace{\baselineskip}

Secondly, there were some technical problems. I was not familiar with programming in Python and didn't have enough knowledge about regressions, pilots, and machine learning topics.

\vspace{\baselineskip}

Thirdly, we sometimes had communication problems and couldn't agree on certain topics. We arranged a meeting with Peter Braun to ask our questions and clarify everything.

\vspace{\baselineskip}

It was challenging to write code in Python. Therefore, I started with implementing the fundamental rules and sought help from my teammates when things became more difficult. Additionally, I had other skills from my past experiences. Apart from implementing rules, I also assisted in the output processes to make our results visible on the website, enabling us to discuss on results and generate a comprehensive report with the total score. Moreover, I had experience with LaTeX, so I contributed to the documentation by embedding my teammates' findings to LaTeX. I attended all meetings, just like my teammates, and frequently expressed my opinions about the tasks and implementation processes.


\subsubsection{Müberra Şeyma Uslu}

At the beginning, it was a bit challenging to decide on which programming language to use, as we were familiar with different languages. Additionally, understanding each other's perspectives was difficult due to our individual coding styles. However, we overcame these communication problems by frequently coming together and having discussions. Through these conversations, we identified each team member's strengths and weaknesses, allowing us to allocate tasks accordingly and proceed with the project in that manner.


\subsubsection{Kemal Eren Öztürk}



\textbf{Communication Problems:}

There were a few instances of communication breakdowns within the team. Sometimes ideas were not communicated clearly, leading to misunderstandings and confusion. This resulted in some inefficiencies and rework. However, we quickly realized the importance of proper communication and made efforts to improve our clarity and actively listen to each other.

\vspace{\baselineskip}


\textbf{Distributing Tasks:}

Initially, all team members attempted to code a single Python program, which proved to be highly ineffective. As a result, we decided to split into two groups, each responsible for a specific aspect of the project. This division significantly improved our efficiency and allowed for smoother progress in both designing and coding.

\vspace{\baselineskip}

\textbf{Design:}

The process of programming each rule was relatively easy and straightforward. However, designing the software architecture posed significant challenges. We struggled with finding the most efficient and effective way to structure the software, which caused delays and confusion. But in the end, we pushed ourselves to design a proper software design.

\vspace{\baselineskip}

\textbf{Procrastination:}

Procrastination became a minor issue during the project. We occasionally found ourselves delaying tasks or not fully utilizing our time. This led to some unnecessary pressure and rushed work towards the end. However, we were able to recognize this problem early on and actively worked on improving our time management and productivity.

\subsection{Lessons learned for future projects}

\subsubsection{Illia Rohalskyi}


I was not very familiar with git, so this really will help me in my future. I also was not aware of structuing the project, and, I guess, I would do it differently. I would think of structure in the very first meeting, instead of reworking everything in the middle of project development.

\vspace{\baselineskip}


 In contrast to my fellows, I believe, Python was a wise choice for this project. Python has a great ecosystem to work with data and is also very simple. In my perspective, if we would use other language (such as Java, Rust or any C-based language), our team would have harder times implementing the project, as those languages are more difficult to learn. We also did not need those extra miliseconds of faster running time. So overall, in my opinion, python was perfect for this task.

\vspace{\baselineskip}

 
I learned how to communicate with people, share ideas and manage team with different backgrounds. I guess, this is the most valuable lesson from this project.


\subsubsection{Kaan Özer}

I learned that there is no shame in restarting everything from the beginning several times in a project because sometimes we might change our minds and need to modify our classes and implementations. 


\vspace{\baselineskip}

 

I learned that Starting a project with a new language that you don't know at all is not a big deal after learning some programming basics and having good teammates. Additionally, I learned many things about machine learning and the Python environment.


\vspace{\baselineskip}


In future projects, I will avoid the mindset of "Let's complete it first, and then we'll organize it," as it often leads to a significant waste of time. Instead, I recognized the importance of organizing everything from scratch to the production, as procrastination can result in a chaotic and troublesome outcome.

\subsubsection{Müberra Şeyma Uslu}

I had never been involved in a team project where we collaborated on coding for such a long period before, so it was a significant plus and learning experience for me. Although I had previous experience with Git, I had not worked extensively with a version control system as a group, which allowed me to learn and utilize Git at a more advanced level. Our goal was to evaluate an existing exam plan based on specific criteria, and since each rule had different evaluation criteria, we had to analyze the available data in various ways. For example, we leveraged regression for one of the rules. That was bit intensive and hard for me but I had fun while working on it. If I were to develop a similar program again, I would opt for a C-based programming language instead of Python.


\subsubsection{Kemal Eren Öztürk}

Don't code the first thing that comes to mind: Rushing into coding without sufficient planning and consideration can lead to inefficiencies and complications later on. It is essential to take the time to properly analyze and design the software architecture before diving into implementation.
Explaining ideas properly to team members is crucial: Clear and effective communication is vital for efficient teamwork. It is essential to ensure that ideas and concepts are communicated clearly to avoid misunderstandings and confusion among team members. Encouraging active listening and providing regular opportunities for clarification can significantly improve collaboration.


