\section{Technical Foundations}

\subsection{Regression basics:}

 In the context of this work, \textit{linear regression} is a technique that aims to find a line that best fits a given set of data points. It does so by minimizing the distance between the line and the data points. The formula of the line is $y = mx + b$, which represents a straight line.

\textit{Polynomial regression} refers to a line that has $N$ number of polynomials in its formula. The formula for polynomial regression is given by $y = M_{n}X^{n} + M_{n-1}X^{n-1} + \ldots + m_{1}x + b$, where $m$ is the slope and $b$ is the intercept. Both the parameters $m$ and $b$ are approximated by regression. The degree $N$ of the polynomial is determined heuristically. Using polynomials enables the line to be non-linear and have some curvature.
